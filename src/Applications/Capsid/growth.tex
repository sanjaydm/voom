\documentclass[
pre,
%%preprint,
 twocolumn,
%% showpacs,
%% preprintnumbers,
amsmath,
amssymb
]{revtex4}
%\documentclass[preprint,showpacs,preprintnumbers,amsmath,amssymb]{revtex4}

\usepackage{times}
\usepackage{graphicx}% Include figure files
\usepackage{dcolumn}% Align table columns on decimal point
\usepackage{bm}% bold math
\usepackage{color}
\usepackage{subfigure}

%% \graphicspath{{./fig/}}

%\nofiles

\begin{document}

\title{An elastic model for growth of spherical and icosahedral viruses}

\author{William S. Klug}
\email{klug@ucla.edu}
\affiliation{%
Department of Mechanical and Aerospace Engineering,
University of California Los Angeles,
Los Angeles, CA 90095-1597 USA
}%

\date{\today}

\begin{abstract}


%% Valid PACS numbers may be entered using the \verb+\pacs{#1}+ command.
\end{abstract}

%% \pacs{Valid PACS appear here}% PACS, the Physics and Astronomy
%%                              % Classification Scheme.

%%\keywords{Suggested keywords}%Use showkeys class option if keyword
                              %display desired
\maketitle

%%
%%--------------------------------------------------------------------
%%
\section{Introduction} 
%%
%%--------------------------------------------------------------------
%%
Here I outline a model, in the style of a Ginzbug-Landau theory, for the growth of an elastic shell onto the surface of a sphere.  Consider a flat, unstressed patch of a thin elastic sheet occupying domain $\bar{\Omega}\subset \mathbb{R}^3$, with boundary $\partial\bar{\Omega}$, with material particles described by reference position $\bar{\bm{x}}\in\bar{\Omega}$.  Physically, this patch is taken to represent the nucleation of a growing capsid shell, for which the capsid proteins are naturally uncurved so that  they form a natural hexahedral tiling of the 2-D plane.  Now let us imagine wrapping this patch onto the surface of a sphere of radius $R$, which I'll denote by $S^2_R\subset\mathbb{R}^3$.  The wrapping is described explicitly in terms of a deformation mapping 
\[
\bm{\varphi}(\bar{\bm{x}}):\bar{\Omega}\to S^2_R
\]
that gives the deformed positions of the material particles as $\bm{x} = \bm{\varphi}(\bar{\bm{x}})$.  The deformed configuration of the sheet now covers a portion of the sphere, denoted $\Omega\equiv \bm{\varphi}(\bar{\Omega})\subset S^2_R$.   For a given reference configuration $\bar{\Omega}$, the elastic energy can be minimized subject to the constraint that the sheet is wrapped onto the sphere, $\bm{x}=\bm{\varphi}(\bar{\bm{x}})\in S^2_R$.  When wrapping the sphere with a circular reference patch, it is intuitively apparent that compressive in-plane strains will persist toward the boundary and tensile strains toward the center.  For more complex patch shapes, it is less clear how the material will be strained.  

We hypothesize that during the process of capsid growth, material will be added to the reference configuration $\bar{\Omega}$ in such a way as to minimize the rate of increase in elastic energy.  In other words, we suppose that grow faster at points on the boundary of the reference configuration $\partial\bar{\Omega}$ where elastic energy is lower, and slower at points where it is higher.  The model presented here is designed as a quantitative test of this hypothesis.

%%
%%--------------------------------------------------------------------
%%

%%
%%--------------------------------------------------------------------
%%
\section{A diffuse boundary model}
%%
%%--------------------------------------------------------------------
%%
The strain energy of the shell is the sum of two
contributions, one arising from bending of the surface, and the second
resulting from inplane stretching \citep{LandauLifshitz,Timoshenko}.  Since we are forcing the sheet to wrap onto a sphere, the bending energy density is a constant term proportional to $1/R^2$ and maybe neglected.
%
The energy of stretching is described in terms of the two-dimensional
Green Strain tensor with covariant components
\begin{equation}
E_{\alpha\beta} = \frac{1}{2}(a_{\alpha\beta} -
\bar{a}_{\alpha\beta}) ,
\end{equation}
where $a_{\alpha\beta}$ and $\bar{a}_{\alpha\beta}$ are deformed and reference surface metrics with respect to a generic set of curvelinear coordinates parametrizing the configurations of the sheet as $\bm{x}(s^1,s^2)$ and $\bar{\bm{x}}(s^1,s^2)$.  The simplest and lowest order (i.e.,
isotropic and linear) stress-strain relations lead to a strain energy
of the form
\begin{equation}\label{eq:strainEnergy}
\mathcal{F}_E = \int_{\bar{\Omega}} \frac{1}{2}%
  \left[
    2\mu_0 E^{\alpha\beta}E_{\alpha\beta} + \lambda_0 (E^\alpha_\alpha)^2
  \right]
  d\bar{S} ,
\end{equation}
where $\lambda_0$ and $\mu_0$ are the two-dimensional Lam\'e constants,
given in terms of the 2-D Young's modulus $E$ and Poisson's ratio
$\nu$ as
\[
\lambda_0 = \frac{E\nu}{(1-\nu^2)} , \qquad \mu_0 = \frac{E}{2(1+\nu)} .
\]
%

To sidestep the difficulties of explicitly parameterizing the reference domain  --- the growth of which is one of the key unknowns for which we wish to solve --- we describe $\bar{\Omega}$ implicitly through the use of an indicator field 
\[
\rho(\bar{\bm{x}}):\mathbb{R}^2\to[0,1] ,
\]
which can be taken to represent a normalized material density function, with $\rho=1$ representing regions containing material particles at their unstressed density, and $\rho=0$ representing regions containing no material particles.  Thus, the reference configuration is defined implicitly as
\[
\bar{\Omega} = \{\bar{\bm{x}}\in\mathbb{R}^2 \ | \ \rho(\bar{\bm{x}}) > 0 \} .
\]
This gives a natural definition of the area of the reference configuration of the sheet as
\[
A(\bar{\Omega}) = \int_{\mathbb{R}^2} \rho d\bar{S} .
\]
Additionally, we now write the elastic energy as
\begin{equation}\label{eq:strainEnergyGL}
\mathcal{F}_E = \int_{\mathbb{R}^2} \rho \frac{1}{2}%
 \left[
    2\mu_0 E^{\alpha\beta}E_{\alpha\beta} + \lambda_0 (E^\alpha_\alpha)^2
  \right]
  d\bar{S} .
\end{equation}
%
Furthermore, to force the indicator field to be either 0 or 1 almost everywhere, we add a double well GL potential $g(\rho)$ with minima at 0 and 1, and we penalize gradients in the indicator field, i.e., 
\begin{equation}\label{eq:GLEnergy}
\mathcal{F}_{GL} = \int_{\mathbb{R}^2} \Big[ g(\rho) + \frac{\Gamma}{2} (\bar{a}^{\alpha\beta}\rho_{,\alpha}\rho_{,\beta}) \Big]
  d\bar{S} .
\end{equation}
Note that the indicator gradient term will produce an effective line tension proportional to the length of the boundary of $\partial\bar{\Omega}$, defined as the region where $\rho$ transitions from 1 to 0.
%
Lastly, to model growth of the sheet we add a term proportional to the reference area
\begin{equation}\label{eq:chemicalPotential}
\mathcal{F}_{A} = -\int_{\mathbb{R}^2} \mu_A \rho \ d\bar{S} ,
\end{equation}
where $\mu_A$ is the chemical potential per unit area for adding new material particles to the sheet.  The total free energy is then a summation of all of the above terms
\[
\mathcal{F} = \mathcal{F}_E + \mathcal{F}_{GL} + \mathcal{F}_A
\]

To model growth of a capsid at constant chemical potential, we evolve the deformation map and indicator functions according to gradient flow as
\begin{subequations}\label{eq:flow}
\begin{eqnarray}
 C^\varphi \frac{\partial \bm{\varphi}}{\partial t} &=& - \frac{\delta \mathcal{F}}{\delta \bm{\varphi}} \\
C^\rho \frac{\partial \rho}{\partial t} &=& -  \frac{\delta \mathcal{F}}{\delta \rho} ,
\end{eqnarray}
\end{subequations}
where kinetic coefficients $C^\varphi$ and $C^\rho$ maybe adjusted independently to model varying the rate of growth relative to the rate of stress relaxation.  These equations are the typical Allen-Cahn type for modeling the kinetics of a diffuse interface for a non-conserved order-parameter.

\subsection{GL Energy and Line Tension} 
The diffuse interface between where the indicator field transitions from $\rho=0$ to $\rho=1$ obtained from minimizing the total energy is characterized by an energy per unit length, or line tension, $\tau$,
\[
\tau = \sqrt{ \Gamma \Delta g} ,
\]
where $\Delta g$ is a measure of the barrier height of the GL potential $g(\rho)$.  The characteristic width of the interface is
\[
\xi = \sqrt{\Gamma / \Delta g} . 
\]
To set an expected range for parameters $\Delta g$ and $\Gamma$, we assume an interface width on the order of the size of an individual protein
\[
\xi \approx 1-2 \text{nm}
\]
and a barrier height of roughly a few $k_BT$ per protein
\[
\Delta g \approx 1-10 \frac{k_BT}{\pi \xi^2} \approx 0.1-10\frac{k_B T}{\text{nm}^2} .
\]
From these we obtain
\[
\tau/\xi = \Delta g
\]
%%
%%--------------------------------------------------------------------
%%
\subsection{Kinetic coefficients}\label{sec:kinetics} %
%%
%%--------------------------------------------------------------------
%%
The simplest choice for modeling the kinetics of the two fields $\bm{\varphi}$ and $\rho$ is to set their kinetic coefficients to be constants.  However, we would like to also explore the effect of varying the rate of evolution of the deformation field in the ordered (near $\rho=1$) and disordered (near $\rho=0$) regions.  To achieve this, we may consider a generic power-series expansion for the corresponding kinetic coefficient
\[
C^\varphi:=\sum_{i=0}^n C^{\varphi,i}\rho^i .
\]
In particular, we consider the linear ($n=1$) case
\[
C^\varphi = C^{\varphi,0} + C^{\varphi,1} \rho .
\]
Physical intuition suggests $C^{\varphi,0}\ge 0$, with equality implying complete relaxation of the strain field in the disordered state where $\rho=0$.  Likewise, it seems reasonable to choose $C^{\varphi,1}\ge 0$, such that the deformation is allowed to relax slower in the ordered state, allowing strain to be ``locked in'' during growth.

%%
%%--------------------------------------------------------------------
%%
\section{\label{sec:FEM}Finite element formulation}
%%
%%--------------------------------------------------------------------
%%

Following the standard weighted residual approach, we derive the weak form functional, with arbitrary admissible weight or test functions $\delta\bm{\varphi}$ and $\delta\rho$
\begin{multline}
G[\bm{\varphi},\rho;\delta\bm{\varphi},\delta\rho] := \\ 
\int_{\mathbb{R}^2} 
\left[
C^\varphi\frac{\partial\bm{\varphi}}{\partial t} + \frac{\delta \mathcal{F}}{\delta\bm{\varphi}} 
\right]\cdot\delta\bm{\varphi} 
d\bar{S} 
+ \int_{\mathbb{R}^2} 
\left[
C^\rho\frac{\partial\rho}{\partial t} + \frac{\delta \mathcal{F}}{\delta\rho}
\right] \delta\rho
d\bar{S} \\
 = \int_{\mathbb{R}^2} 
\left[
C^\varphi\frac{\partial\bm{\varphi}}{\partial t}\cdot\delta\bm{\varphi}  +
C^\rho\frac{\partial\rho}{\partial t} \delta\rho 
\right]
d\bar{S} + \delta\mathcal{F}
\end{multline}
The weak form of the BVP follows by demanding that the trial functions $(\bm{\varphi},\rho)$ satisfy $G[\bm{\varphi},\rho;\delta\bm{\varphi},\delta\rho]=0$ for all admissible test functions  $(\delta\bm{\varphi},\delta\rho)$.

We next discretize the 2-D domain into a mesh of polygonal subdomains
(finite elements), defining the reference position map over each of these elements
by piecewise polynomial interpolation among a set of nodal points $\{\bar{\bm{x}}_a, a=1,\dots,N\}$ as
\begin{equation}\label{eq:referenceInterpolation}
  \bar{\bm{x}}(s^1,s^2) = \sum_{a=1}^\mathcal{N}{\bar{\bm{x}}_{a} N_{a}(s^{1},s^{2}) } ,
\end{equation}
where the piecewise polynomial functions
$N_a(s^1,s^2)$ (shape functions) combine to form a globally $C^0$
conforming position map.  
%
We adopt an isoparametric formulation in which the deformation map and the indicator field are interpolated using these same shape functions, among deformed nodal positions $\bm{x}_a$ and nodal indicator values $\rho_a$,
\begin{equation}\label{eq:deformedInterpolation}
  \bm{x}(s^1,s^2) = \bm{\varphi}(\bar{\bm{x}}) = \sum_{a=1}^\mathcal{N}{\bm{x}_{a} N_{a}(s^{1},s^{2}) } , 
\end{equation}
%
\begin{equation}\label{eq:indicatorInterpolation}
  \rho(s^1,s^2) = \sum_{a=1}^\mathcal{N}{\rho_a N_{a}(s^{1},s^{2}) } .
\end{equation}
Likewise, following the Galerkin appoach, we use the same interpolation for the test functions
\begin{equation}
  \delta\bm{\varphi} = \sum_{a=1}^\mathcal{N}{\delta\bm{x}_{a} N_{a} } , \quad
  \delta\rho = \sum_{a=1}^\mathcal{N}{\delta\rho_a N_{a} } ,
\end{equation}
%
with arbitrary nodal weights  $\delta\bm{x}_a$ and $\delta\rho_a$.

Insertion of these interpolation equations into the weak form gives a set of semi-discrete Allen-Cahn time-evolution equations for the nodal variables
\begin{subequations}\label{eq:FEflow}
\begin{eqnarray}
\sum_{b=1}^N C^\varphi_{ab}\dot{\bm{x}}_b &=& -  \frac{\partial \mathcal{F}}{\partial \bm{x}_a} \\
\sum_{b=1}^N C^\rho_{ab}\dot{\rho_b} &=& - \frac{\partial \mathcal{F}}{\partial \rho_a} ,
\end{eqnarray}
\end{subequations}
The terms on the right hand sided can be interpreted as nodal internal forces and nodal chemical potentials, and are given explicitly as
\begin{equation*}
\bm{f}_{a} = \frac{\partial \mathcal{F}}{\partial \bm{x}_{a}} = \int \bm{n}^\alpha N_{a,\alpha} d\bar{S} 
\end{equation*}
\begin{equation*}
\mu_{a} = \frac{\partial \mathcal{F}}{\partial \rho_{a}} = \int \mu N_a + \lambda^\alpha N_{a,\alpha} d\bar{S} 
\end{equation*}
where 
\begin{equation}
\bm{n}^\alpha = \sigma^{\alpha\beta}\bm{a}_\beta = \rho\big[2\mu_0 E^{\alpha\beta} + \lambda_0(\text{tr }\mathbf{E})\bar{a}^{\alpha\beta}\big]\bm{a}_\beta,
\end{equation}
are the stress resultants, and 
\begin{eqnarray*}
\mu &=& \frac{1}{2}%
 \left[
    2\mu_0 E^{\alpha\beta}E_{\alpha\beta} + \lambda_0 (E^\alpha_\alpha)^2
 \right] 
+ g' - \mu_A \\
\lambda^\alpha &=& \Gamma \bar{a}^{\alpha\beta}\rho_{,\beta} 
\end{eqnarray*}
are chemical potential resultants.
%
The terms on the left-hand side of eq.(\ref{eq:FEflow}) are interpreted as generalized ``drag'' forces, with 
\[
C^\varphi_{ab} = \int C^\varphi   N_a N_b \; d\bar{S}, \quad
C^\rho_{ab} = \int C^\rho N_a N_b \; d\bar{S} 
\]
denoting the components of nodal ``drag'' matrices.  In practice, these may be approximated by row-sum lumping, $C^{(\cdot)}_{a}:=\sum_{b=1}^N C^{(\cdot)}_{ab}$, yielding a diagonal or decoupled system of nodal equations
\begin{subequations}\label{eq:diagFEflow}
\begin{eqnarray}
C^\varphi_{a}\dot{\bm{x}}_a &=& -  \bm{f}_a \\
C^\rho_{a}\dot{\rho}_a &=& - \mu_a \quad\text{(no sum on $a$)}.
\end{eqnarray}
\end{subequations}
Note that because the finite-element shape functions form a partition of unity ($\sum_{b=1}^N N_b = 1$), the lumped nodal drags are simply
\[
C^\varphi_{a} = \int C^\varphi  N_a \; d\bar{S}, \quad
C^\rho_{a} = \int C^\rho  N_a \; d\bar{S} 
\]

%%
%%--------------------------------------------------------------------
%%
\section{Time integration}\label{sec:time} %
%%
%%--------------------------------------------------------------------
%%
To evolve the semi-discrete nodal equations in time, we construct a sequence of time steps $t^0, t^1, \dots$, with $t^{n+1}=t^n+\Delta t$.  The nodal field values at time step $t^n$ are denoted $\bm{x}_a^{(n)}$ and $\rho_a^{(n)}$.  To step from a solution at $t^n$ to $t^{n+1}$, we consider both explicit (forward) Euler and implicit (backward) Euler integration schemes.
 
\textit{Explicit Scheme:} The Forward Euler approximation for time derivatives $ \dot{f}^{(n)} \to (f^{(n+1)}-f^{(n)})/\Delta t$
yields
\begin{eqnarray*}
\sum_{b}C_{ab}^{\varphi}(t^{n})\frac{\bm{x}_b^{(n+1)}-\bm{x}_b^{(n)}}{\Delta t} &=& -\bm{f}_a(t^{n}) \\
\sum_{b}C_{ab}^{\rho}(t^{n})\frac{\rho_b^{(n+1)}-\rho_b^{(n)}}{\Delta t} &=& -\mu_a(t^{n})  .
\end{eqnarray*}
%
Writing the update explicitly,
\begin{eqnarray*}
\bm{x}_a^{(n+1)}&=&\bm{x}_a^{(n)} -\Delta t \sum_{b}\big[C_{ab}(t^{n}) \big]^{-1}\bm{f}_b(t^n) \\
\rho_a^{(n+1)}&=&\rho_a^{(n)} - \Delta t \sum_{b}\big[C_{ab}(t^{n}) \big]^{-1}\mu_b(t^n)  
\end{eqnarray*}
This scheme is obviously most effiecient when diagonal or lumped drags are used.

\textit{Semi-implicit Scheme:}  We insert the  Backward Euler approximation for time derivatives $ \dot{f}^{(n)} \to (f^{(n)}-f^{(n-1)})/\Delta t$ into the nodal equations while evaluating the drag matrices at the previous time-step, to yield
\begin{eqnarray*}
\sum_{b}C_{ab}^{\varphi}(t^{n})\frac{\bm{x}_b^{(n+1)}-\bm{x}_b^{(n)}}{\Delta t} &=& -\bm{f}_a(t^{n+1})  \\
\sum_{b}C_{ab}^{\rho}(t^{n})\frac{\rho_b^{(n+1)}-\rho_b^{(n)}}{\Delta t} &=& -\mu_a(t^{n+1})  
\end{eqnarray*}
%
which is a system of nonlinear equations that must be solved iteratively for the unknowns at $t^{n+1}$.  To facilitate the use of generic optimization solvers, it is helpful to note this is equivalent to minimizing the following function
\begin{multline*}
I(\bm{x}^{(n+1)},\rho^{(n+1)}) = \mathcal{F}(\bm{x}^{(n+1)},\rho^{(n+1)}) \\ 
+ \sum_{a,b}\frac{ C^\varphi_{ab}(t^{n})}{2\Delta t}\big[\bm{x}_a^{(n+1)}-\bm{x}_a^{(n)}\big]\cdot\big[\bm{x}_b^{(n+1)}-\bm{x}_b^{(n)}\big] \\
+ \sum_{a,b}\frac{ C^\rho_{ab}(t^{n})}{2\Delta t}\big[\rho_a^{(n+1)}-\rho_a^{(n)}\big]\big[\rho_b^{(n+1)}-\rho_b^{(n)}\big] .
\end{multline*}
\newpage
%%
%%--------------------------------------------------------------------
%%
\appendix
%%
%%--------------------------------------------------------------------
%%

%%
%%--------------------------------------------------------------------
%%
\section{Kinematics of a curved surface}\label{sec:kinematics} %
%%
%%--------------------------------------------------------------------
%%
Details regarding the geometry of surfaces can be found in any of the
standard texts on differential geometry (e.g., \citep{Sokolnikoff,
DoCarmo}).  Here I review the essential relations necessary for
development of the finite element approximation. 

Let $\bar{\mathcal{S}}\subset\mathbb{R}^3$ represent a shell surface
in a (stress-free) reference state, and
$\mathcal{S}\subset\mathbb{R}^3$ the surface of the same shell in a
deformed configuration.  Both of these may be parameterized by a set
of curvilinear coordinates $\{s^{1}, s^{2}\}$, such that the position
vector on the reference and deformed configurations is given by the
maps $\bar{\bm{x}}(s^{1}, s^{2})$ and $\bm{x}(s^{1}, s^{2})$,
respectively.  The (covariant) basis vectors corresponding to the
curvilinear coordinates $(s^\alpha)$ on the reference and deformed
surfaces are
%
%
\begin{equation}\label{eq:basis-vector}
  \bar{\bm{a}}_\alpha = \frac{\partial \bar{\bm{x}}}{\partial s^\alpha} 
  \equiv \bar{\bm{x}}_{,\alpha} , \quad
  \bm{a}_\alpha = \frac{\partial \bm{x}}{\partial s^\alpha} 
  \equiv \bm{x}_{,\alpha} .
\end{equation}
%
The summation convention is implied with Greek indices taking values
of 1 and 2, and comma is used to denote partial differentiation with
respect to the surface coordinates here as well as in the rest of this
paper. Subsequent derivations deal only with the deformed
configuration, it being understood that reference quantities with an
overbar are analogous.  

The dual (contravariant) basis vectors $\bm{a}^{\alpha}$ are defined
such that $\bm{a}^{\alpha} \cdot \bm{a}_{\beta} =
\delta_{\beta}^{\alpha}$.
%
%\begin{equation}
%  \bm{a}^{\alpha} \cdot \bm{a}_{\beta} = \delta_{\beta}^{\alpha} ,
%\end{equation}
%
The covariant and contravariant components of the surface metric
tensor in turn follow as
%
\begin{equation}
  a_{\alpha \beta} = \bm{a}_\alpha \cdot \bm{a}_\beta , \qquad
  a^{\alpha \beta} = \bm{a}^\alpha \cdot \bm{a}^\beta ,
\end{equation}
respectively.
%
The infinitesimal element of area over the surface $\mathcal{S}$
at point ($s^{1}, s^{2}$) is then
%
\begin{equation}
  dA = \sqrt{a} \, ds^{1}ds^{2} \equiv \sqrt{a} d^{2}s ,
\end{equation}
%
where $\sqrt{a}$ is the surface measure and $a$ is the determinant of
the covariant metric tensor
%
%% \begin{equation}
\(
  a = | a_{\alpha\beta} | .
\)
%% \end{equation}
%
Two-dimensional strain is related to local length changes on the
surface.  If $d\bar{\bm{x}}$ and $d\bm{x}$ are the vectors connecting
two infinitesimally separated points on the surface, respectively
before and after deformation, then the difference of the squares of their magnitudes can be written
\begin{eqnarray*}
ds^2 - d\bar{s}^2 &=& d\bm{x}\cdot d\bm{x} - d\bar{\bm{x}}\cdot d\bar{\bm{x}}\\
 &=& (\bm{a}_\alpha ds^\alpha)\cdot(\bm{a}_\beta ds^\beta) 
 - (\bar{\bm{a}}_\alpha ds^\alpha)\cdot(\bar{\bm{a}}_\beta ds^\beta)\\
 &=& (a_{\alpha\beta} - \bar{a}_{\alpha\beta}) ds^\alpha ds^\beta\\
 &=& 2 E_{\alpha\beta} ds^\alpha ds^\beta
\end{eqnarray*}
where $E_{\alpha\beta}$ are defined as the covariant components of the
Green Strain tensor
\begin{equation}
\mathbf{E} = \frac{1}{2}(a_{\alpha\beta} -
\bar{a}_{\alpha\beta})\bar{\bm{a}}^\alpha \otimes \bar{\bm{a}}^\beta .
\end{equation}

Bending of the shell is related to changes in the shell director, defined as the unit normal to the surface,
%
\begin{equation} 
\bm{d} \equiv \bm{a}_{3} 
= \frac{\bm{a}_1 \times \bm{a}_2}{|\bm{a}_1\times \bm{a}_2|} 
= \frac{\bm{a}_1 \times \bm{a}_2}{\sqrt{a}} . 
\end{equation}
%
The symmetric curvature tensor or shape operator $\mathbf{S}$ is
defined having covariant components
\begin{equation}\label{eq:curvature-tensor}
  b_{\alpha\beta}  = -\bm{d}_{,\alpha}\cdot\bm{a}_{\beta} = \bm{d} \cdot \bm{a}_{\alpha,\beta} = b_{\beta\alpha}.
\end{equation}
Bending is most naturally described in terms of two invariants of the
curvature tensor, the mean curvature and Gaussian curvature.  The mean
curvature is one half of the trace of the curvature tensor
\begin{equation}\label{eq:mean-curvature-1}
  H = \frac{S}{2}; \quad S = \text{tr }\mathbf{S} =
  b^{\alpha}_{\alpha} = -\bm{a}^{\alpha} \cdot
  \bm{d}_{,\alpha} ,
\end{equation}
%or in terms of basis vectors
%\begin{equation}\label{eq:mean-curvature-2}
%  H = -\frac{1}{2}\bm{a}^{\alpha} \cdot \bm{d}_{,\alpha}
%\end{equation}
and the Gaussian curvature is the determinant of the curvature
tensor
\begin{equation}\label{eq:G-curvature}
  K = \det\mathbf{S} = | b^{\alpha}_{\beta} | = \frac{b}{a} ,
\end{equation}
where $b = |b_{\alpha\beta}|$ is the determinant of the matrix of
covariant curvature components.


\vfill

%% \newpage %Just because of unusual number of tables stacked at end
%% \bibliography{ref}% Produces the bibliography via BibTeX.

\end{document}
%
% ****** End of file apssamp.tex ******
